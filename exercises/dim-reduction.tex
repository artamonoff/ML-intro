% !TEX root = exercises-ml.tex

\section{Снижение размерности}

\begin{exercise}
Загрузите датасет \texttt{Labour}.
\begin{enumerate}
	\item Визуализируйте данные в главных компонентах (рассмотрите 2D и 3D визуализацию)
	\item Визуализируйте данные, используя метод t-SNE (рассмотрите 2D и 3D визуализацию)
	\item Вычислите накопленные дисперсии главных компонент.
\end{enumerate}
\end{exercise}

\begin{exercise}
В условиях предыдущей задачи проведите визуализацию и вычислите накопленные дисперсии главных компонент
после (нелинейного) преобразования данных (квантильное, Box-Cox, Yeo-Johnson)
\end{exercise}

\begin{exercise}
Загрузите датасет \texttt{sleep75} и \textbf{удалите переменные с пропущенными значениями}.
\begin{enumerate}
	\item Визуализируйте данные в главных компонентах (рассмотрите 2D и 3D визуализацию)
	\item Визуализируйте данные, используя метод t-SNE (рассмотрите 2D и 3D визуализацию)
	\item Вычислите накопленные дисперсии главных компонент.
\end{enumerate}
\end{exercise}
	
\begin{exercise}
В условиях предыдущей задачи проведите визуализацию и вычислите накопленные дисперсии главных компонент
после (нелинейного) преобразования данных (квантильное, Box-Cox, Yeo-Johnson)
\end{exercise}

\begin{exercise}
Загрузите датасет \texttt{diamonds} и \textbf{удалите категориальные переменные}.
\begin{enumerate}
	\item Визуализируйте данные в главных компонентах (рассмотрите 2D и 3D визуализацию)
	\item Визуализируйте данные, используя метод t-SNE (рассмотрите 2D и 3D визуализацию)
	\item Вычислите накопленные дисперсии главных компонент.
\end{enumerate}
\end{exercise}
	
\begin{exercise}
В условиях предыдущей задачи проведите визуализацию и вычислите накопленные дисперсии главных компонент
после (нелинейного) преобразования данных (квантильное, Box-Cox, Yeo-Johnson)
\end{exercise}