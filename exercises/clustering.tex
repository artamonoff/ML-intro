% !TEX root = exercises-ml.tex

\section{Кластеризация}

\textbf{Важно} \textit{обязательно проводим предварительную обработку данных:
\begin{itemize}
	\item удаление пропущенных значений
	\item нормировка
	\item преобразование категориальных признаков
\end{itemize}
}

\begin{exercise}
Для набора данных \texttt{countries} проведите разбиение на кластеры следующими
методам:
\begin{center}
	\begin{tabular}{c|l}
		Число кластеров & Метод \\ \hline
		3 & k-средних \\
		4 & k-средних \\
		5 & k-средних \\
		3 & иерархическая \\
		4 & иерархическая \\
		5 & иерархическая \\ \hline
	\end{tabular}
\end{center}
Визуализируйте разбиение на кластеры на диаграмме рассеяния в переменных датасета
\end{exercise}

\begin{exercise}
Для набора данных \texttt{countries} найдите <<оптимальное>> число кластеров
для метода
\begin{enumerate}
	\item k-средних
	\item иерархической кластеризации
\end{enumerate}
относительно метрик: Silhouette, Calinski-Harabasz, Davies-Bouldin
\end{exercise}

\begin{exercise}
Из набора данных \texttt{sleep75} возьмите переменные 
\texttt{sleep, totwrk, age, educ} и проведите разбиение на кластеры следующими
методам:
\begin{center}
	\begin{tabular}{c|l}
		Число кластеров & Метод \\ \hline
		3 & k-средних \\
		4 & k-средних \\
		5 & k-средних \\
		3 & иерархическая \\
		4 & иерархическая \\
		5 & иерархическая \\ \hline
	\end{tabular}
\end{center}
Визуализируйте разбиение на кластеры на диаграмме рассеяния в переменных датасета
\end{exercise}
	
\begin{exercise}
Из набора данных \texttt{sleep75} возьмите переменные 
\texttt{sleep, totwrk, age, educ} и найдите <<оптимальное>> число кластеров
для метода
\begin{enumerate}
	\item k-средних
	\item иерархической кластеризации
\end{enumerate}
относительно метрик: Silhouette, Calinski-Harabasz, Davies-Bouldin
\end{exercise}

\begin{exercise}
Для набора данных \texttt{Labour} проведите разбиение на кластеры следующими
методам:
\begin{center}
	\begin{tabular}{c|l}
		Число кластеров & Метод \\ \hline
		3 & k-средних \\
		4 & k-средних \\
		5 & k-средних \\
		3 & иерархическая \\
		4 & иерархическая \\
		5 & иерархическая \\ \hline
	\end{tabular}
\end{center}
Визуализируйте разбиение на кластеры на диаграмме рассеяния в переменных датасета
\end{exercise}
	
\begin{exercise}
Для набора данных \texttt{Labour} найдите <<оптимальное>> число кластеров
для метода
\begin{enumerate}
	\item k-средних
	\item иерархической кластеризации
\end{enumerate}
относительно метрик: Silhouette, Calinski-Harabasz, Davies-Bouldin
\end{exercise}