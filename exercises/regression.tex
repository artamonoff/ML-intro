% !TEX root = exercises-ml.tex

\section{Регрессия}

\subsection{k-NN}

\textbf{Важной}: \textit{каждой задаче раздела модель нужно обучить на полном датасете}

\begin{exercise}
Для набора данных \texttt{sleep75} рассмотрим задачу прогнозирования
для переменных
\begin{center}
	\begin{tabular}{|c|c|}\hline
		зависимая/target & объясняющая/предикторы/features \\ \hline
		sleep & totwrk, age, south, male \\ \hline
	\end{tabular}
\end{center}
\begin{enumerate}
	\item подгоните на исходном датасете модель k-NN с параметрами
	\begin{center}
		\begin{tabular}{|l|l|l|}\hline
		№ & \(k\) & веса \\ \hline
		1 & 5 & uniform \\
		2 & 5 & distance \\
		3 & 10 & uniform \\
		4 & 10 & distance \\ \hline
		\end{tabular}
	\end{center}
	\item Рассмотрим трёх людей с характеристиками
	\begin{center}
		\begin{tabular}{|l||l|l|l|l|}\hline
			index & totwrk & age & south & male \\ \hline\hline
			0 & 2160 & 32 & 1 & 0 \\
			1 & 1720 & 24 & 0 & 1 \\
			2 & 2390 & 44 & 0 & 1 \\ \hline
		\end{tabular}
	\end{center}
	вычислите прогноз \textbf{sleep} по каждой модели
\end{enumerate}
\end{exercise}

\begin{exercise}
Для набора данных \texttt{sleep75} рассмотрим задачу прогнозирования
для переменных
\begin{center}
	\begin{tabular}{|c|c|}\hline
		зависимая/target & объясняющая/предикторы/features \\ \hline
		sleep & totwrk, age, south, male, smsa, yngkid, marr \\ \hline
	\end{tabular}
\end{center}
\begin{enumerate}
	\item подгоните на исходном датасете модель k-NN с параметрами
	\begin{center}
		\begin{tabular}{|l|l|l|}\hline
		№ & \(k\) & веса \\ \hline
		1 & 5 & uniform \\
		2 & 5 & distance \\
		3 & 10 & uniform \\
		4 & 10 & distance \\ \hline
		\end{tabular}
	\end{center}
	\item Рассмотрим трёх людей с характеристиками
	\begin{center}
		\begin{tabular}{|l||l|l|l|l|l|l|l|}\hline
			index & totwrk & age & south & male & smsa & yngkid & marr \\ \hline\hline
			0 & 2150 & 37 & 0 & 1 & 1 & 0 & 1 \\
			1 & 1950 & 28 & 1 & 1 & 0 & 1 & 0 \\
			2 & 2240 & 26 & 0 & 0 & 1 & 0 & 0 \\ \hline
		\end{tabular}
	\end{center}
	вычислите прогноз \textbf{sleep} по каждой модели
\end{enumerate}
\end{exercise}

\begin{exercise}
Для набора данных \texttt{wage2} рассмотрим задачу прогнозирования
для переменных
\begin{center}
	\begin{tabular}{|c|c|}\hline
		зависимая/target & объясняющая/предикторы/features \\ \hline
		wage & age, IQ, south, married, urban \\ \hline
	\end{tabular}
\end{center}
\begin{enumerate}
	\item подгоните на исходном датасете модель k-NN с параметрами
	\begin{center}
		\begin{tabular}{|l|l|l|}\hline
		№ & \(k\) & веса \\ \hline
		1 & 5 & uniform \\
		2 & 5 & distance \\
		3 & 10 & uniform \\
		4 & 10 & distance \\ \hline
		\end{tabular}
	\end{center}
	\item Рассмотрим трёх людей с характеристиками
	\begin{center}
		\begin{tabular}{|l||l|l|l|l|l|}\hline
			index & age & IQ & south & married & urban \\ \hline\hline
			0 & 36 & 105 & 1 & 1 & 1 \\
			1 & 29 & 123 & 0 & 1 & 0 \\
			2 & 25 & 112 & 1 & 0 & 1 \\ \hline
		\end{tabular}
	\end{center}
	вычислите прогноз \textbf{wage} по каждой модели
\end{enumerate}
\end{exercise}

\begin{exercise}
Для набора данных \texttt{wage2} рассмотрим задачу прогнозирования
для переменных
\begin{center}
	\begin{tabular}{|c|c|}\hline
		зависимая/target & объясняющая/предикторы/features \\ \hline
		log(wage) & age, IQ, south, married, urban \\ \hline
	\end{tabular}
\end{center}
\begin{enumerate}
	\item подгоните на исходном датасете модель k-NN с параметрами
	\begin{center}
		\begin{tabular}{|l|l|l|}\hline
		№ & \(k\) & веса \\ \hline
		1 & 5 & uniform \\
		2 & 5 & distance \\
		3 & 10 & uniform \\
		4 & 10 & distance \\ \hline
		\end{tabular}
	\end{center}
	\item Рассмотрим трёх людей с характеристиками
	\begin{center}
		\begin{tabular}{|l||l|l|l|l|l|}\hline
			index & age & IQ & south & married & urban \\ \hline\hline
			0 & 36 & 105 & 1 & 1 & 1 \\
			1 & 29 & 123 & 0 & 1 & 0 \\
			2 & 25 & 112 & 1 & 0 & 1 \\ \hline
		\end{tabular}
	\end{center}
	вычислите прогноз \textbf{wage} по каждой модели
\end{enumerate}
\end{exercise}

\begin{exercise}
Для набора данных \texttt{wage1} рассмотрим задачу прогнозирования
для переменных
\begin{center}
	\begin{tabular}{|c|c|}\hline
		зависимая/target & объясняющая/предикторы/features \\ \hline
		wage & exper, female, married, smsa \\ \hline
	\end{tabular}
\end{center}
\begin{enumerate}
	\item подгоните на исходном датасете модель k-NN с параметрами
	\begin{center}
		\begin{tabular}{|l|l|l|}\hline
		№ & \(k\) & веса \\ \hline
		1 & 5 & uniform \\
		2 & 5 & distance \\
		3 & 10 & uniform \\
		4 & 10 & distance \\ \hline
		\end{tabular}
	\end{center}
	\item Рассмотрим трёх людей с характеристиками
	\begin{center}
		\begin{tabular}{|l||l|l|l|l|}\hline
			index & exper & female & married & smsa \\ \hline\hline
			0 & 5 & 1 & 1 & 1  \\
			1 & 26 & 0 & 0 & 1 \\
			2 & 38 & 1 & 1 & 0 \\ \hline
		\end{tabular}
	\end{center}
	вычислите прогноз \textbf{wage} по каждой модели
\end{enumerate}
\end{exercise}

\begin{exercise}
Для набора данных \texttt{wage1} рассмотрим задачу прогнозирования
для переменных
\begin{center}
	\begin{tabular}{|c|c|}\hline
		зависимая/target & объясняющая/предикторы/features \\ \hline
		log(wage) & exper, female, married, smsa \\ \hline
	\end{tabular}
\end{center}
\begin{enumerate}
	\item подгоните на исходном датасете модель k-NN с параметрами
	\begin{center}
		\begin{tabular}{|l|l|l|}\hline
		№ & \(k\) & веса \\ \hline
		1 & 5 & uniform \\
		2 & 5 & distance \\
		3 & 10 & uniform \\
		4 & 10 & distance \\ \hline
		\end{tabular}
	\end{center}
	\item Рассмотрим трёх людей с характеристиками
	\begin{center}
		\begin{tabular}{|l||l|l|l|l|}\hline
			index & exper & female & married & smsa \\ \hline\hline
			0 & 5 & 1 & 1 & 1  \\
			1 & 26 & 0 & 0 & 1 \\
			2 & 38 & 1 & 1 & 0 \\ \hline
		\end{tabular}
	\end{center}
	вычислите прогноз \textbf{wage} по каждой модели
\end{enumerate}
\end{exercise}

\begin{exercise}
Для набора данных \texttt{Labour} рассмотрим задачу прогнозирования
для переменных
\begin{center}
	\begin{tabular}{|c|c|}\hline
		зависимая/target & объясняющая/предикторы/features \\ \hline
		output & capital, labour \\ \hline
	\end{tabular}
\end{center}
\begin{enumerate}
	\item подгоните на исходном датасете модель k-NN с параметрами
	\begin{center}
		\begin{tabular}{|l|l|l|}\hline
		№ & \(k\) & веса \\ \hline
		1 & 5 & uniform \\
		2 & 5 & distance \\
		3 & 10 & uniform \\
		4 & 10 & distance \\ \hline
		\end{tabular}
	\end{center}
	\item Рассмотрим трёх людей с характеристиками
	\begin{center}
		\begin{tabular}{|l||l||l|l|}\hline
			index & capital & labour \\ \hline\hline
			0 & 2.970 & 85 \\
			1 & 10.450 & 60  \\
			2 & 3.850 & 105 \\ \hline
		\end{tabular}
	\end{center}
	вычислите прогноз \textbf{output} по каждой модели
\end{enumerate}
\end{exercise}

\begin{exercise}
Для набора данных \texttt{Labour} рассмотрим задачу прогнозирования
для переменных
\begin{center}
	\begin{tabular}{|c|c|}\hline
		зависимая/target & объясняющая/предикторы/features \\ \hline
		log(output) & log(capital), log(labour) \\ \hline
	\end{tabular}
\end{center}
\begin{enumerate}
	\item подгоните на исходном датасете модель k-NN с параметрами
	\begin{center}
		\begin{tabular}{|l|l|l|}\hline
		№ & \(k\) & веса \\ \hline
		1 & 5 & uniform \\
		2 & 5 & distance \\
		3 & 10 & uniform \\
		4 & 10 & distance \\ \hline
		\end{tabular}
	\end{center}
	\item Рассмотрим трёх людей с характеристиками
	\begin{center}
		\begin{tabular}{|l||l||l|l|}\hline
			index & capital & labour \\ \hline\hline
			0 & 2.970 & 85 \\
			1 & 10.450 & 60  \\
			2 & 3.850 & 105 \\ \hline
		\end{tabular}
	\end{center}
	вычислите прогноз \textbf{output} по каждой модели
\end{enumerate}
\end{exercise}

\begin{exercise}
Для набора данных \texttt{Labour} рассмотрим задачу прогнозирования
для переменных
\begin{center}
	\begin{tabular}{|c|c|}\hline
		зависимая/target & объясняющая/предикторы/features \\ \hline
		output & capital, labour, wage \\ \hline
	\end{tabular}
\end{center}
\begin{enumerate}
	\item подгоните на исходном датасете модель k-NN с параметрами
	\begin{center}
		\begin{tabular}{|l|l|l|}\hline
		№ & \(k\) & веса \\ \hline
		1 & 5 & uniform \\
		2 & 5 & distance \\
		3 & 10 & uniform \\
		4 & 10 & distance \\ \hline
		\end{tabular}
	\end{center}
	\item Рассмотрим трёх людей с характеристиками
	\begin{center}
		\begin{tabular}{|l||l|l|l|}\hline
			index & capital & labour & wage \\ \hline\hline
			0 & 2.970 & 85 & 36.98\\
			1 & 10.450 & 60 & 33.82  \\
			2 & 3.850 & 105 & 40.23\\ \hline
		\end{tabular}
	\end{center}
	вычислите прогноз \textbf{output} по каждой модели
\end{enumerate}
\end{exercise}

\begin{exercise}
Для набора данных \texttt{Labour} рассмотрим задачу прогнозирования
для переменных
\begin{center}
	\begin{tabular}{|c|c|}\hline
		зависимая/target & объясняющая/предикторы/features \\ \hline
		log(output) & log(capital), log(labour), log(wage) \\ \hline
	\end{tabular}
\end{center}
\begin{enumerate}
	\item подгоните на исходном датасете модель k-NN с параметрами
	\begin{center}
		\begin{tabular}{|l|l|l|}\hline
		№ & \(k\) & веса \\ \hline
		1 & 5 & uniform \\
		2 & 5 & distance \\
		3 & 10 & uniform \\
		4 & 10 & distance \\ \hline
		\end{tabular}
	\end{center}
	\item Рассмотрим трёх людей с характеристиками
	\begin{center}
		\begin{tabular}{|l||l|l|l|}\hline
			index & capital & labour & wage \\ \hline\hline
			0 & 2.970 & 85 & 36.98\\
			1 & 10.450 & 60 & 33.82  \\
			2 & 3.850 & 105 & 40.23\\ \hline
		\end{tabular}
	\end{center}
	вычислите прогноз \textbf{output} по каждой модели
\end{enumerate}
\end{exercise}

\subsection{Линейная регрессия}

\textbf{Важной}: \textit{каждой задаче раздела модель нужно обучить на полном датасете}

\begin{exercise}
Для набора данных \texttt{sleep75} рассмотрим линейную регрессию 
\begin{center}
	\textbf{sleep на totwrk, age, south, male}.
\end{center}
\begin{enumerate}
	\item Подгоните модель
	\begin{itemize}
		\item без регуляризации
		\item с регуляризацией Ridge (\(\alpha=1\))
		\item с регуляризацией LASSO (\(\alpha=1\))
	\end{itemize}
	и выведите коэффициенты подогнанной модели
	\item Рассмотрим трёх людей с характеристиками
	\begin{center}
		\begin{tabular}{|l||l|l|l|l|}\hline
			index & totwrk & age & south & male \\ \hline\hline
			0 & 2160 & 32 & 1 & 0 \\
			1 & 1720 & 24 & 0 & 1 \\
			2 & 2390 & 44 & 0 & 1 \\ \hline
		\end{tabular}
	\end{center}
	вычислите прогноз \textbf{sleep} по каждому методу подгонки
	% \item На обучающей выборке вычислите метрики подгонки: \(R^2\), 
	% MSE, MAE, MAPE, RMSE
\end{enumerate}
\end{exercise}

\begin{exercise}
Для набора данных \texttt{sleep75} рассмотрим линейную регрессию 
\begin{center}
	\textbf{sleep на totwrk, age, south, male, smsa, yngkid, marr}.
\end{center}
\begin{enumerate}
	\item Подгоните модель
	\begin{itemize}
		\item без регуляризации
		\item с регуляризацией Ridge (\(\alpha=1\))
		\item с регуляризацией LASSO (\(\alpha=1\))
	\end{itemize}
	и выведите коэффициенты подогнанной модели
	\item Рассмотрим трёх людей с характеристиками
	\begin{center}
		\begin{tabular}{|l||l|l|l|l|l|l|l|}\hline
			index & totwrk & age & south & male & smsa & yngkid & marr \\ \hline\hline
			0 & 2150 & 37 & 0 & 1 & 1 & 0 & 1 \\
			1 & 1950 & 28 & 1 & 1 & 0 & 1 & 0 \\
			2 & 2240 & 26 & 0 & 0 & 1 & 0 & 0 \\ \hline
		\end{tabular}
	\end{center}
	вычислите прогноз \textbf{sleep} по каждому методу подгонки
	% \item На обучающей выборке вычислите метрики подгонки: \(R^2\), 
	% MSE, MAE, MAPE, RMSE
\end{enumerate}
\end{exercise}

\begin{exercise}
Для набора данных \texttt{wage2} рассмотрим линейную регрессию 
\begin{center}
	\textbf{wage на age, IQ, south, married, urban}.
\end{center}
\begin{enumerate}
	\item Подгоните модель
	\begin{itemize}
		\item без регуляризации
		\item с регуляризацией Ridge (\(\alpha=1\))
		\item с регуляризацией LASSO (\(\alpha=1\))
	\end{itemize}
	и выведите коэффициенты подогнанной модели
	\item Рассмотрим трёх людей с характеристиками
	\begin{center}
		\begin{tabular}{|l||l|l|l|l|l|}\hline
			index & age & IQ & south & married & urban \\ \hline\hline
			0 & 36 & 105 & 1 & 1 & 1 \\
			1 & 29 & 123 & 0 & 1 & 0 \\
			2 & 25 & 112 & 1 & 0 & 1 \\ \hline
		\end{tabular}
	\end{center}
	вычислите прогноз \textbf{wage} по каждому методу подгонки
	% \item На обучающей выборке вычислите метрики подгонки: \(R^2\), 
	% MSE, MAE, MAPE, RMSE
\end{enumerate}
\end{exercise}

\begin{exercise}
Для набора данных \texttt{wage2} рассмотрим линейную регрессию 
\begin{center}
	\textbf{log(wage) на age, IQ, south, married, urban}.
\end{center}
\begin{enumerate}
	\item Подгоните модель
	\begin{itemize}
		\item без регуляризации
		\item с регуляризацией Ridge (\(\alpha=1\))
		\item с регуляризацией LASSO (\(\alpha=1\))
	\end{itemize}
	и выведите коэффициенты подогнанной модели
	\item Рассмотрим трёх людей с характеристиками
	\begin{center}
		\begin{tabular}{|l||l|l|l|l|l|}\hline
			index & age & IQ & south & married & urban \\ \hline\hline
			0 & 36 & 105 & 1 & 1 & 1 \\
			1 & 29 & 123 & 0 & 1 & 0 \\
			2 & 25 & 112 & 1 & 0 & 1 \\ \hline
		\end{tabular}
	\end{center}
	вычислите прогноз \textbf{wage} по каждому методу подгонки
	% \item На обучающей выборке вычислите метрики подгонки: \(R^2\), 
	% MSE, MAE, MAPE, RMSE
\end{enumerate}
\end{exercise}

\begin{exercise}
Для набора данных \texttt{wage1} рассмотрим линейную регрессию 
\begin{center}
	\textbf{wage на exper, female, married, smsa}.
\end{center}
\begin{enumerate}
	\item Подгоните модель
	\begin{itemize}
		\item без регуляризации
		\item с регуляризацией Ridge (\(\alpha=1\))
		\item с регуляризацией LASSO (\(\alpha=1\))
	\end{itemize}
	и выведите коэффициенты подогнанной модели
	\item Рассмотрим трёх людей с характеристиками
	\begin{center}
		\begin{tabular}{|l||l|l|l|l|}\hline
			index & exper & female & married & smsa \\ \hline\hline
			0 & 5 & 1 & 1 & 1  \\
			1 & 26 & 0 & 0 & 1 \\
			2 & 38 & 1 & 1 & 0 \\ \hline
		\end{tabular}
	\end{center}
	вычислите прогноз \textbf{wage} по каждому методу подгонки
	% \item На обучающей выборке вычислите метрики подгонки: \(R^2\), 
	% MSE, MAE, MAPE, RMSE
\end{enumerate}
\end{exercise}

\begin{exercise}
Для набора данных \texttt{wage1} рассмотрим линейную регрессию 
\begin{center}
	\textbf{log(wage) на exper, female, married, smsa}.
\end{center}
\begin{enumerate}
	\item Подгоните модель
	\begin{itemize}
		\item без регуляризации
		\item с регуляризацией Ridge (\(\alpha=1\))
		\item с регуляризацией LASSO (\(\alpha=1\))
	\end{itemize}
	и выведите коэффициенты подогнанной модели
	\item Рассмотрим трёх людей с характеристиками
	\begin{center}
		\begin{tabular}{|l||l|l|l|l|}\hline
			index & exper & female & married & smsa \\ \hline\hline
			0 & 5 & 1 & 1 & 1  \\
			1 & 26 & 0 & 0 & 1 \\
			2 & 38 & 1 & 1 & 0 \\ \hline
		\end{tabular}
	\end{center}
	вычислите прогноз \textbf{wage} по каждому методу подгонки
	% \item На обучающей выборке вычислите метрики подгонки: \(R^2\), 
	% MSE, MAE, MAPE, RMSE
\end{enumerate}
\end{exercise}

\begin{exercise}
Для набора данных \texttt{Labour} рассмотрим линейную регрессию 
\begin{center}
	\textbf{output на capital, labour}.
\end{center}
\begin{enumerate}
	\item Подгоните модель
	\begin{itemize}
		\item без регуляризации
		\item с регуляризацией Ridge (\(\alpha=1\))
		\item с регуляризацией LASSO (\(\alpha=1\))
	\end{itemize}
	и выведите коэффициенты подогнанной модели
	\item Рассмотрим три фирмы с характеристиками
	\begin{center}
		\begin{tabular}{|l||l||l|l|}\hline
			index & capital & labour \\ \hline\hline
			0 & 2.970 & 85 \\
			1 & 10.450 & 60  \\
			2 & 3.850 & 105 \\ \hline
		\end{tabular}
	\end{center}
	вычислите прогноз \textbf{output} по каждому методу подгонки
	% \item На обучающей выборке вычислите метрики подгонки: \(R^2\), 
	% MSE, MAE, MAPE, RMSE
\end{enumerate}
\end{exercise}

\begin{exercise}
Для набора данных \texttt{Labour} рассмотрим линейную регрессию 
\begin{center}
	\textbf{log(output) на log(capital), log(labour)}.
\end{center}
\begin{enumerate}
	\item Подгоните модель
	\begin{itemize}
		\item без регуляризации
		\item с регуляризацией Ridge (\(\alpha=1\))
		\item с регуляризацией LASSO (\(\alpha=1\))
	\end{itemize}
	и выведите коэффициенты подогнанной модели
	\item Рассмотрим три фирмы с характеристиками
	\begin{center}
		\begin{tabular}{|l||l|l|}\hline
			index & capital & labour \\ \hline\hline
			0 & 2.970 & 85 \\
			1 & 10.450 & 60  \\
			2 & 3.850 & 105 \\ \hline
		\end{tabular}
	\end{center}
	вычислите прогноз \textbf{output} по каждому методу подгонки
	% \item На обучающей выборке вычислите метрики подгонки: \(R^2\), 
	% MSE, MAE, MAPE, RMSE
\end{enumerate}
\end{exercise}

\begin{exercise}
Для набора данных \texttt{Labour} рассмотрим линейную регрессию 
\begin{center}
	\textbf{output на capital, labour, wage}.
\end{center}
\begin{enumerate}
	\item Подгоните модель
	\begin{itemize}
		\item без регуляризации
		\item с регуляризацией Ridge (\(\alpha=1\))
		\item с регуляризацией LASSO (\(\alpha=1\))
	\end{itemize}
	и выведите коэффициенты подогнанной модели
	\item Рассмотрим три фирмы с характеристиками
	\begin{center}
		\begin{tabular}{|l||l|l|l|}\hline
			index & capital & labour & wage \\ \hline\hline
			0 & 2.970 & 85 & 36.98\\
			1 & 10.450 & 60 & 33.82  \\
			2 & 3.850 & 105 & 40.23\\ \hline
		\end{tabular}
	\end{center}
	вычислите прогноз \textbf{output} по каждому методу подгонки
	% \item На обучающей выборке вычислите метрики подгонки: \(R^2\), 
	% MSE, MAE, MAPE, RMSE
\end{enumerate}
\end{exercise}

\begin{exercise}
Для набора данных \texttt{Labour} рассмотрим линейную регрессию 
\begin{center}
	\textbf{log(output) на log(capital), log(labour), log(wage)}.
\end{center}
\begin{enumerate}
	\item Подгоните модель
	\begin{itemize}
		\item без регуляризации
		\item с регуляризацией Ridge (\(\alpha=1\))
		\item с регуляризацией LASSO (\(\alpha=1\))
	\end{itemize}
	и выведите коэффициенты подогнанной модели
	\item Рассмотрим три фирмы с характеристиками
	\begin{center}
		\begin{tabular}{|l||l|l|l|}\hline
			index & capital & labour & wage \\ \hline\hline
			0 & 2.970 & 85 & 36.98\\
			1 & 10.450 & 60 & 33.82  \\
			2 & 3.850 & 105 & 40.23\\ \hline
		\end{tabular}
	\end{center}
	вычислите прогноз \textbf{output} по каждому методу подгонки
	% \item На обучающей выборке вычислите метрики подгонки: \(R^2\), 
	% MSE, MAE, MAPE, RMSE
\end{enumerate}
\end{exercise}


\subsection{Валидация моделей}

\begin{exercise}
Набор данных \texttt{sleep75} разбейте на обучающую и тестовую часть
в соотношении 80:20.

Рассмотрим задачу прогнозирования для переменных
\begin{center}
	\begin{tabular}{|c|c|}\hline
		зависимая/target & объясняющая/предикторы/features \\ \hline
		sleep & totwrk, age, south, male \\ \hline
	\end{tabular}
\end{center}
и следующие модели
\begin{center}
	\begin{tabular}{|l|l|}\hline
		№ & Модель \\ \hline
		1 & линейная регрессия\\
		2 & k-NN с \(k=5\), веса 'uniform' \\
		3 & k-NN с \(k=5\), веса 'distance' \\
		4 & k-NN с \(k=10\), веса 'uniform' \\
		5 & k-NN с \(k=10\), веса 'distance' \\ \hline
	\end{tabular}
\end{center}
Проведите валидацию моделей относительно метрик \(R^2\), MSE, MAE,
MAPE. Какая модель предпочтительней?
\end{exercise}

\begin{exercise}
Набор данных \texttt{sleep75} разбейте на обучающую и тестовую часть
в соотношении 80:20.

Рассмотрим задачу прогнозирования для переменных
\begin{center}
	\begin{tabular}{|c|c|}\hline
		зависимая/target & объясняющая/предикторы/features \\ \hline
		sleep & totwrk, age, south, male, smsa, yngkid, marr \\ \hline
	\end{tabular}
\end{center}
и следующие модели
\begin{center}
	\begin{tabular}{|l|l|}\hline
		№ & Модель \\ \hline
		1 & линейная регрессия\\
		2 & k-NN с \(k=5\), веса 'uniform' \\
		3 & k-NN с \(k=5\), веса 'distance' \\
		4 & k-NN с \(k=10\), веса 'uniform' \\
		5 & k-NN с \(k=10\), веса 'distance' \\ \hline
	\end{tabular}
\end{center}
Проведите валидацию моделей относительно метрик \(R^2\), MSE, MAE,
MAPE. Какая модель предпочтительней?
\end{exercise}

\begin{exercise}
Набор данных \texttt{wage2} разбейте на обучающую и тестовую часть
в соотношении 80:20.

Рассмотрим задачу прогнозирования для переменных
\begin{center}
	\begin{tabular}{|c|c|}\hline
		зависимая/target & объясняющая/предикторы/features \\ \hline
		wage & age, IQ, south, married, urban \\ \hline
	\end{tabular}
\end{center}
и следующие модели
\begin{center}
	\begin{tabular}{|l|l|}\hline
		№ & Модель \\ \hline
		1 & линейная регрессия\\
		2 & k-NN с \(k=5\), веса 'uniform' \\
		3 & k-NN с \(k=5\), веса 'distance' \\
		4 & k-NN с \(k=10\), веса 'uniform' \\
		5 & k-NN с \(k=10\), веса 'distance' \\ \hline
	\end{tabular}
\end{center}
Проведите валидацию моделей относительно метрик \(R^2\), MSE, MAE,
MAPE. Какая модель предпочтительней?
\end{exercise}

\begin{exercise}
Набор данных \texttt{wage2} разбейте на обучающую и тестовую часть
в соотношении 80:20.

Рассмотрим задачу прогнозирования для переменных
\begin{center}
	\begin{tabular}{|c|c|}\hline
		зависимая/target & объясняющая/предикторы/features \\ \hline
		log(wage) & age, IQ, south, married, urban \\ \hline
	\end{tabular}
\end{center}
и следующие модели
\begin{center}
	\begin{tabular}{|l|l|}\hline
		№ & Модель \\ \hline
		1 & линейная регрессия\\
		2 & k-NN с \(k=5\), веса 'uniform' \\
		3 & k-NN с \(k=5\), веса 'distance' \\
		4 & k-NN с \(k=10\), веса 'uniform' \\
		5 & k-NN с \(k=10\), веса 'distance' \\ \hline
	\end{tabular}
\end{center}
Проведите валидацию моделей относительно метрик \(R^2\), MSE, MAE,
MAPE. Какая модель предпочтительней?
\end{exercise}

\begin{exercise}
Набор данных \texttt{wage1} разбейте на обучающую и тестовую часть
в соотношении 80:20.

Рассмотрим задачу прогнозирования для переменных
\begin{center}
	\begin{tabular}{|c|c|}\hline
		зависимая/target & объясняющая/предикторы/features \\ \hline
		wage & exper, female, married, smsa \\ \hline
	\end{tabular}
\end{center}
и следующие модели
\begin{center}
	\begin{tabular}{|l|l|}\hline
		№ & Модель \\ \hline
		1 & линейная регрессия\\
		2 & k-NN с \(k=5\), веса 'uniform' \\
		3 & k-NN с \(k=5\), веса 'distance' \\
		4 & k-NN с \(k=10\), веса 'uniform' \\
		5 & k-NN с \(k=10\), веса 'distance' \\ \hline
	\end{tabular}
\end{center}
Проведите валидацию моделей относительно метрик \(R^2\), MSE, MAE,
MAPE. Какая модель предпочтительней?
\end{exercise}

\begin{exercise}
Набор данных \texttt{wage1} разбейте на обучающую и тестовую часть
в соотношении 80:20.

Рассмотрим задачу прогнозирования для переменных
\begin{center}
	\begin{tabular}{|c|c|}\hline
		зависимая/target & объясняющая/предикторы/features \\ \hline
		log(wage) & exper, female, married, smsa \\ \hline
	\end{tabular}
\end{center}
и следующие модели
\begin{center}
	\begin{tabular}{|l|l|}\hline
		№ & Модель \\ \hline
		1 & линейная регрессия\\
		2 & k-NN с \(k=5\), веса 'uniform' \\
		3 & k-NN с \(k=5\), веса 'distance' \\
		4 & k-NN с \(k=10\), веса 'uniform' \\
		5 & k-NN с \(k=10\), веса 'distance' \\ \hline
	\end{tabular}
\end{center}
Проведите валидацию моделей относительно метрик \(R^2\), MSE, MAE,
MAPE. Какая модель предпочтительней?
\end{exercise}

\begin{exercise}
Набор данных \texttt{Labour} разбейте на обучающую и тестовую часть
в соотношении 80:20.

Рассмотрим задачу прогнозирования для переменных
\begin{center}
	\begin{tabular}{|c|c|}\hline
		зависимая/target & объясняющая/предикторы/features \\ \hline
		output & capital, labour, wage \\ \hline
	\end{tabular}
\end{center}
и следующие модели
\begin{center}
	\begin{tabular}{|l|l|}\hline
		№ & Модель \\ \hline
		1 & линейная регрессия\\
		2 & k-NN с \(k=5\), веса 'uniform' \\
		3 & k-NN с \(k=5\), веса 'distance' \\
		4 & k-NN с \(k=10\), веса 'uniform' \\
		5 & k-NN с \(k=10\), веса 'distance' \\ \hline
	\end{tabular}
\end{center}
Проведите валидацию моделей относительно метрик \(R^2\), MSE, MAE,
MAPE. Какая модель предпочтительней?
\end{exercise}

\begin{exercise}
Набор данных \texttt{Labour} разбейте на обучающую и тестовую часть
в соотношении 80:20.

Рассмотрим задачу прогнозирования для переменных
\begin{center}
	\begin{tabular}{|c|c|}\hline
		зависимая/target & объясняющая/предикторы/features \\ \hline
		log(output) & log(capital), log(labour), log(wage) \\ \hline
	\end{tabular}
\end{center}
и следующие модели
\begin{center}
	\begin{tabular}{|l|l|}\hline
		№ & Модель \\ \hline
		1 & линейная регрессия\\
		2 & k-NN с \(k=5\), веса 'uniform' \\
		3 & k-NN с \(k=5\), веса 'distance' \\
		4 & k-NN с \(k=10\), веса 'uniform' \\
		5 & k-NN с \(k=10\), веса 'distance' \\ \hline
	\end{tabular}
\end{center}
Проведите валидацию моделей относительно метрик \(R^2\), MSE, MAE,
MAPE. Какая модель предпочтительней?
\end{exercise}