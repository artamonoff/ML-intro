% !TEX root = exercises-ml.tex

\section{Классификация}

\subsection{k-NN}

\begin{exercise}
Для набора данных \texttt{sleep75} рассмотрим переменные
\begin{center}
	\begin{tabular}{|c|c|} \hline
		Зависимая/таргетная & объясняющие/признаки \\ \hline
		male & sleep, totwrk, age, south \\ \hline
	\end{tabular}
\end{center}
Рассмотрим трёх людей с характеристиками
\begin{center}
	\begin{tabular}{|l||l|l|l|l|}\hline
		index & sleep & totwrk & age & south  \\ \hline\hline
		0 & 2900 & 2160 & 32 & 1  \\
		1 & 3120 & 1720 & 24 & 0  \\
		2 & 2850 & 2390 & 44 & 0  \\ \hline
	\end{tabular}
\end{center}
Постройте прогноз для \textbf{male} методом k-NN с параметрами
\begin{center}
	\begin{tabular}{|l|l|l|}\hline
	№ & \(k\) & веса \\ \hline
	1 & 5 & uniform \\
	2 & 5 & distance \\
	3 & 10 & uniform \\
	4 & 10 & distance \\ \hline
	\end{tabular}
\end{center}
\end{exercise}

\begin{exercise}
Для набора данных \texttt{sleep75} рассмотрим переменные
\begin{center}
	\begin{tabular}{|c|c|} \hline
		Зависимая/таргетная & объясняющие/признаки \\ \hline
		smsa & sleep, totwrk, age, south, male, yngkid, marr \\ \hline
	\end{tabular}
\end{center}
Рассмотрим трёх людей с характеристиками
\begin{center}
	\begin{tabular}{|l||l|l|l|l|l|l|l|}\hline
		index & sleep & totwrk & age & south & male & yngkid & marr \\ \hline\hline
		0 & 2900 & 2150 & 37 & 0 & 1 &  0 & 1 \\
		1 & 3120 & 1950 & 28 & 1 & 1 &  1 & 0 \\
		2 & 2850 & 2240 & 26 & 0 & 0 &  0 & 0 \\ \hline
	\end{tabular}
\end{center}
Постройте прогноз для \textbf{smsa} методом k-NN с параметрами
\begin{center}
	\begin{tabular}{|l|l|l|}\hline
	№ & \(k\) & веса \\ \hline
	1 & 5 & uniform \\
	2 & 5 & distance \\
	3 & 10 & uniform \\
	4 & 10 & distance \\ \hline
	\end{tabular}
\end{center}
\end{exercise}

\begin{exercise}
Для набора данных \texttt{default} рассмотрим переменные
\begin{center}
	\begin{tabular}{|c|c|} \hline
		Зависимая/таргетная & объясняющие/признаки \\ \hline
		default & age, income, ownrent, selfempl \\ \hline
	\end{tabular}
\end{center}
Рассмотрим трёх людей с характеристиками
	\begin{center}
		\begin{tabular}{|l||l|l|l|l|}\hline
			index & age & income & ownrent & selfempl  \\ \hline\hline
			0 & 37 & 2000 & 0 & 1  \\
			1 & 42.5 & 5250 & 1 & 0  \\
			2 & 29 & 2916 & 0 & 0  \\ \hline
		\end{tabular}
	\end{center}
Постройте прогноз для \textbf{default} методом k-NN с параметрами
\begin{center}
	\begin{tabular}{|l|l|l|}\hline
	№ & \(k\) & веса \\ \hline
	1 & 5 & uniform \\
	2 & 5 & distance \\
	3 & 10 & uniform \\
	4 & 10 & distance \\ \hline
	\end{tabular}
\end{center}
\end{exercise}

\subsection{Линейные модели}

\begin{exercise}
Для набора данных \texttt{sleep75} рассмотрим переменные

\begin{center}
	\begin{tabular}{|c|c|} \hline
		Зависимая/таргетная & объясняющие/признаки \\ \hline
		male & sleep, totwrk, age, south \\ \hline
	\end{tabular}
\end{center}
\begin{enumerate}
	\item Подгоните логистическую регрессию и выведите коэффициенты подогнанной модели
	\item Рассмотрим трёх людей с характеристиками
	\begin{center}
		\begin{tabular}{|l||l|l|l|l|}\hline
			index & sleep & totwrk & age & south  \\ \hline\hline
			0 & 2900 & 2160 & 32 & 1  \\
			1 & 3120 & 1720 & 24 & 0  \\
			2 & 2850 & 2390 & 44 & 0  \\ \hline
		\end{tabular}
	\end{center}
	вычислите прогноз для \textbf{male} 
\end{enumerate}
\end{exercise}

\begin{exercise}
Для набора данных \texttt{sleep75} рассмотрим переменные

\begin{center}
	\begin{tabular}{|c|c|} \hline
		Зависимая/таргетная & объясняющие/признаки \\ \hline
		male & sleep, totwrk, age, south \\ \hline
	\end{tabular}
\end{center}

\begin{enumerate}
	\item Подгоните модель SVM и выведите коэффициенты подогнанной модели
	\item Рассмотрим трёх людей с характеристиками
	\begin{center}
		\begin{tabular}{|l||l|l|l|l|}\hline
			index & sleep & totwrk & age & south  \\ \hline\hline
			0 & 2900 & 2160 & 32 & 1  \\
			1 & 3120 & 1720 & 24 & 0  \\
			2 & 2850 & 2390 & 44 & 0  \\ \hline
		\end{tabular}
	\end{center}
	вычислите прогноз для \textbf{male} 
\end{enumerate}
\end{exercise}

\begin{exercise}
Для набора данных \texttt{sleep75} рассмотрим переменные
\begin{center}
	\begin{tabular}{|c|c|} \hline
		Зависимая/таргетная & объясняющие/признаки \\ \hline
		smsa & sleep, totwrk, age, south, male, yngkid, marr \\ \hline
	\end{tabular}
\end{center}
\begin{enumerate}
	\item Подгоните логистическую регрессию и выведите коэффициенты подогнанной модели
	\item Рассмотрим трёх людей с характеристиками
	\begin{center}
		\begin{tabular}{|l||l|l|l|l|l|l|l|}\hline
			index & sleep & totwrk & age & south & male & yngkid & marr \\ \hline\hline
			0 & 2900 & 2150 & 37 & 0 & 1 &  0 & 1 \\
			1 & 3120 & 1950 & 28 & 1 & 1 &  1 & 0 \\
			2 & 2850 & 2240 & 26 & 0 & 0 &  0 & 0 \\ \hline
		\end{tabular}
	\end{center}
	вычислите прогноз для \textbf{smsa} 
\end{enumerate}
\end{exercise}

\begin{exercise}
Для набора данных \texttt{sleep75} рассмотрим переменные

\begin{center}
	\begin{tabular}{|c|c|} \hline
		Зависимая/таргетная & объясняющие/признаки \\ \hline
		smsa & sleep, totwrk, age, south, male, yngkid, marr \\ \hline
	\end{tabular}
\end{center}

\begin{enumerate}
	\item Подгоните модель SVM и выведите коэффициенты подогнанной модели
	\item Рассмотрим трёх людей с характеристиками
	\begin{center}
		\begin{tabular}{|l||l|l|l|l|l|l|l|}\hline
			index & sleep & totwrk & age & south & male & yngkid & marr \\ \hline\hline
			0 & 2900 & 2150 & 37 & 0 & 1 &  0 & 1 \\
			1 & 3120 & 1950 & 28 & 1 & 1 &  1 & 0 \\
			2 & 2850 & 2240 & 26 & 0 & 0 &  0 & 0 \\ \hline
		\end{tabular}
	\end{center}
	вычислите прогноз для \textbf{smsa} 
\end{enumerate}
\end{exercise}

\begin{exercise}
Для набора данных \texttt{default} рассмотрим переменные

\begin{center}
	\begin{tabular}{|c|c|} \hline
		Зависимая/таргетная & объясняющие/признаки \\ \hline
		default & age, income, ownrent, selfempl \\ \hline
	\end{tabular}
\end{center}

\begin{enumerate}
	\item Подгоните логистическую регрессию и выведите коэффициенты подогнанной модели
	\item Рассмотрим трёх людей с характеристиками
	\begin{center}
		\begin{tabular}{|l||l|l|l|l|}\hline
			index & age & income & ownrent & selfempl  \\ \hline\hline
			0 & 37 & 2000 & 0 & 1  \\
			1 & 42.5 & 5250 & 1 & 0  \\
			2 & 29 & 2916 & 0 & 0  \\ \hline
		\end{tabular}
	\end{center}
	вычислите прогноз для \textbf{default} 
\end{enumerate}
\end{exercise}

\begin{exercise}
Для набора данных \texttt{default} рассмотрим переменные

\begin{center}
	\begin{tabular}{|c|c|} \hline
		Зависимая/таргетная & объясняющие/признаки \\ \hline
		default & age, income, ownrent, selfempl \\ \hline
	\end{tabular}
\end{center}

\begin{enumerate}
	\item Подгоните модель SVM и выведите коэффициенты подогнанной модели
	\item Рассмотрим трёх людей с характеристиками
	\begin{center}
		\begin{tabular}{|l||l|l|l|l|}\hline
			index & age & income & ownrent & selfempl  \\ \hline\hline
			0 & 37 & 2000 & 0 & 1  \\
			1 & 42.5 & 5250 & 1 & 0  \\
			2 & 29 & 2916 & 0 & 0  \\ \hline
		\end{tabular}
	\end{center}
	вычислите прогноз для \textbf{default} 
\end{enumerate}
\end{exercise}
	
\subsection{Валидация моделей}

\begin{exercise}
Набор данных \texttt{sleep75} разбейте на обучающую и тестовую часть
в соотношении 80:20.

Рассмотрим задачу прогнозирования для переменных
\begin{center}
	\begin{tabular}{|c|c|}\hline
		зависимая/target & объясняющая/предикторы/features \\ \hline
		male & sleep, totwrk, age, south \\ \hline
	\end{tabular}
\end{center}
и следующие модели
\begin{center}
	\begin{tabular}{|l|l|}\hline
		№ & Модель \\ \hline
		1 & Логистическая регрессия\\
		2 & SVM \\ 
		3 & k-NN с \(k=5\), веса 'uniform' \\
		4 & k-NN с \(k=5\), веса 'distance' \\
		5 & k-NN с \(k=10\), веса 'uniform' \\
		6 & k-NN с \(k=10\), веса 'distance' \\ \hline
	\end{tabular}
\end{center}
Проведите валидацию моделей относительно метрик Accuracy, ROC. 
Какая модель предпочтительней?
\end{exercise}

\begin{exercise}
Набор данных \texttt{sleep75} разбейте на обучающую и тестовую часть
в соотношении 80:20.

Рассмотрим задачу прогнозирования для переменных
\begin{center}
	\begin{tabular}{|c|c|}\hline
		зависимая/target & объясняющая/предикторы/features \\ \hline
		smsa & sleep, totwrk, age, south, male, yngkid, marr \\ \hline
	\end{tabular}
\end{center}
и следующие модели
\begin{center}
	\begin{tabular}{|l|l|}\hline
		№ & Модель \\ \hline
		1 & Логистическая регрессия\\
		2 & SVM \\ 
		3 & k-NN с \(k=5\), веса 'uniform' \\
		4 & k-NN с \(k=5\), веса 'distance' \\
		5 & k-NN с \(k=10\), веса 'uniform' \\
		6 & k-NN с \(k=10\), веса 'distance' \\ \hline
	\end{tabular}
\end{center}
Проведите валидацию моделей относительно метрик Accuracy, ROC. Какая модель предпочтительней?
\end{exercise}

\begin{exercise}
Набор данных \texttt{default} разбейте на обучающую и тестовую часть
в соотношении 80:20.
	
Рассмотрим задачу прогнозирования для переменных
\begin{center}
	\begin{tabular}{|c|c|}\hline
		зависимая/target & объясняющая/предикторы/features \\ \hline
		smsa & sleep, totwrk, age, south, male, yngkid, marr \\ \hline
	\end{tabular}
\end{center}
и следующие модели
\begin{center}
	\begin{tabular}{|l|l|}\hline
		№ & Модель \\ \hline
		1 & Логистическая регрессия\\
		2 & SVM \\ 
		3 & k-NN с \(k=5\), веса 'uniform' \\
		4 & k-NN с \(k=5\), веса 'distance' \\
		5 & k-NN с \(k=10\), веса 'uniform' \\
		6 & k-NN с \(k=10\), веса 'distance' \\ \hline
	\end{tabular}
\end{center}
Проведите валидацию моделей относительно метрик Accuracy, ROC. Какая модель предпочтительней?
\end{exercise}